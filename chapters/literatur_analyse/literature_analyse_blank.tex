\subsection{Titel der analysierten Publikation}
\begin{description}[leftmargin=!,labelwidth=\widthof{\bfseries \cite{Be02}}]
	\item [\cite{Be02}] {Vorname Name: Titel der analysierten Publikation, weitere Angaben.}
\end{description}

\textbf{Inhalte}
Was sind die zentralen Inhalte (Themen, interessante Aussagen, Botschaften, Fragestellungen), die in der Arbeit (d.h., in der analysierten Literatur) behandelt werden?
\begin{description}[leftmargin=!,labelwidth=\widthof{\bfseries [(I1)]}]
	\item [(I1)] {...}
	\item [(I2)] {...}
	\item [(I3)] {...}
\end{description}

\textbf{Defizite} 
Welche Defizite bestehender Arbeiten und Lösungen werden als Motivation der eigenen Lösungen genannt?
\begin{description}[leftmargin=!,labelwidth=\widthof{\bfseries [(I1)]}]
	\item [(D1)] {...}
	\item [(D2)] {...}
	\item [(D3)] {...}
\end{description}

\textbf{Prämissen}
Welche Einschränkungen und Vorgaben werden hinsichtlich der eigenen Lösungen ge-macht?
\begin{description}[leftmargin=!,labelwidth=\widthof{\bfseries [(I1)]}]
	\item [(P1)] {...}
	\item [(P2)] {...}
	\item [(P3)] {...}
\end{description}

\textbf{Lösungen}
Was sind die eigenen Lösungen?
\begin{description}[leftmargin=!,labelwidth=\widthof{\bfseries [(I1)]}]
	\item [(L1)] {...}
	\item [(L2)] {...}
	\item [(L3)] {...}
\end{description}

\textbf{Nachweise}
Welche Nachweise (Evidence) werden hinsichtlich der Tragfähigkeit der eigenen Lösun-gen geliefert?
\begin{description}[leftmargin=!,labelwidth=\widthof{\bfseries [(I1)]}]
	\item [(N1)] {...}
	\item [(N2)] {...}
	\item [(N3)] {...}
\end{description}

\textbf{Offene Fragen} 
Welche Fragen sind noch ungelöst geblieben bzw. stellen sich dem Leser?
\begin{description}[leftmargin=!,labelwidth=\widthof{\bfseries [(I1)]}]
	\item [(O1)] {...}
	\item [(O2)] {...}
	\item [(O3)] {...}
\end{description}

\textbf{Sonstiges} \\
Punkte, die in keine der oben genannten Kategorien passen
