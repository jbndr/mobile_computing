
%==============================================================================
% Dokument einrichten und Pakete laden
%==============================================================================

\documentclass[
	pdftex,				% PDFTex verwenden da ausschliesslich ein PDF erzeugt wird.
	a4paper, twoside,	% Verwenden von DIN A4 Papier.
	11pt,				% Grosse Schrift, besser geeignet für A4.
	parskip=half,		% Halbe Zeile Abstand zwischen Absätzen.
	numbers=noenddot,	% Keine Punkte hinter Nummern
	pagesize,           % Schreibt die Papiergroesse in die Datei 
	BCOR10mm,			% Bindekorrektur
	DIV13,				% Alternativ 12 oder 14
	headinclude,		% Kopfzeile in den Textbereich
	headsepline,		% Linie nach Kopfzeile.
	titlepage,
	headings=small,
	bibliography=totocnumbered,	% Bibliographie im TOC nummeriert
]{scrbook}

\usepackage{setspace}\setstretch{1.2} 
\usepackage{titlesec}

\titlespacing\section{0pt}{12pt plus 4pt minus 2pt}{0pt plus 2pt minus 2pt}
\titlespacing\subsection{0pt}{12pt plus 4pt minus 2pt}{0pt plus 2pt minus 2pt}
\titlespacing\subsubsection{0pt}{12pt plus 4pt minus 2pt}{0pt plus 2pt minus 2pt}

\usepackage{scrhack}

\usepackage[printonlyused]{acronym}
\usepackage{multicol}

%
% Zeichenkodieruung und Sprache
%

\usepackage[utf8]{inputenc} % Dokument
\usepackage[T1]{fontenc} % Schrift
\usepackage[ngerman]{babel}

%
% PDF-Metadaten setzen
%

\usepackage{pdfpages}
\usepackage[
	% Titel des PDF Dokuments
	pdftitle={Berücksichtigung von Skalierbarkeitsaspekten bei der Entwicklung von Microservice-basierten Anwendungen},
	% Autor des PDF Dokumentsba_
	pdfauthor={Johannes Binder},
	% Thema des PDF Dokuments
	pdfsubject={},
	% Erzeuger des PDF Dokuments
	pdfcreator={},
	% Schlüsselwörter für das PDF
	pdfkeywords={},
	% Dokumenttitel statt Dateiname anzeigen
	pdfdisplaydoctitle=true,																% Sprache des Dokuments
	pdflang=de,
	bookmarksopen=true,
	bookmarksdepth=1,
	colorlinks,
	linkcolor = black,
	citecolor=black,
	urlcolor=black,
]{hyperref}

%
%  Zusätzliche Pakete laden
%

% Anführungszeichen
\usepackage[style=american]{csquotes}

% erweiterte Tabelleneigenschaftn
\usepackage{array}

% Einbinden von Grafiken
\usepackage{graphicx}	

% mathematischer Textsatz
%\usepackage{amsmath}
%\usepackage{amssymb}
%\usepackage{dsfont}

% Textteile drehen
%\usepackage{rotating}	

% Farbpakete
%\usepackage{color}
%\usepackage{xcolor}

% Quellcode sauber formatieren
\usepackage{listings}	

% Font 'Latin Modern Family' verwenden
\usepackage{microtype}
\usepackage{helvet}
\usepackage{mathptmx}

%==============================================================================
% Einstellungen und Definitionen
%==============================================================================

% Farben definieren

\definecolor{light-gray}{gray}{0.95}
%\definecolor{LinkColor}{rgb}{0,0,0.5}
%\definecolor{ListingBackground}{rgb}{0.85,0.85,0.85}
%\definecolor{CommentColor}{rgb}{0, 0.5, 0}
%\definecolor{StringColor}{rgb}{0.63, 0.09, 0.09}

% KOMA-Script Option, Zeilenumbruch bei Bildbeschreibungen.
\setcapindent{1em}

% Stil der Kopf- und Fusszeilen.
\usepackage[headsepline,automark,komastyle, nouppercase]{scrpage2}
\pagestyle{scrheadings}

% Stil der Überschriften auf normale Schrift.

%\setkomafont{sectioning}{\normalfont\bfseries}		 % Titel mit Normalschrift
\setkomafont{captionlabel}{\normalfont\bfseries}	 % Fette Beschriftungen 
%\setkomafont{pageheadfoot}{\normalfont\itshape}     % Kursive Seitentitel
\setkomafont{descriptionlabel}{\normalfont\bfseries} % Fette Beschreibungstitel

%==============================================================================
% Listings
%==============================================================================

\lstloadlanguages{% Check Dokumentation for further languages ...
  XML,
  HTML,
  Java,
  Tex
}


\lstset{
  %basicstyle=\scriptsize\ttfamily, % Standardschrift
	basicstyle=\footnotesize\ttfamily,
  %numbers=left, % Ort der Zeilennummern
  %numberstyle=\tiny, % Stil der Zeilennummern
  %stepnumber=2, % Abstand zwischen den Zeilennummern
	%numberblanklines=false,
  numbersep=5pt, % Abstand der Nummern zum Text
  tabsize=2, % Groesse von Tabs
  extendedchars=true, %
  breaklines=true, % Zeilen werden Umgebrochen
  %keywordstyle=\color{red},
  frame=b,
  % keywordstyle=[1]\textbf, % Stil der Keywords
  % keywordstyle=[2]\textbf, %
  % keywordstyle=[3]\textbf, %
  % keywordstyle=[4]\textbf, \sqrt{\sqrt{}} %
  %stringstyle=\color{white}\ttfamily, % Farbe der String
  showspaces=false, % Leerzeichen anzeigen ?
  showtabs=false, % Tabs anzeigen ?
  xleftmargin=17pt,
	xrightmargin=17pt,
  framexleftmargin=17pt,
  framexrightmargin=17pt,
  framexbottommargin=4pt,
  backgroundcolor=\color{white},
  showstringspaces=false % Leerzeichen in Strings anzeigen ?
}

\lstset{literate=%
    {Ö}{{\"O}}1
    {Ä}{{\"A}}1
    {Ü}{{\"U}}1
    {ß}{{\ss}}1
    {ü}{{\"u}}1 
    {ä}{{\"a}}1
    {ö}{{\"o}}1
    {~}{{\textasciitilde}}1
}

\usepackage{caption}
\DeclareCaptionFont{white}{\color{white}}
\DeclareCaptionFormat{listing}{\colorbox[cmyk] {0.43, 0.35, 0.35,0.01}{\parbox{\textwidth-2\fboxsep-2\fboxrule-0pt} {\hspace{15pt}#1#2#3}}}
\captionsetup[lstlisting]{format=listing,labelfont=white, textfont=white,singlelinecheck=false, margin=0pt, font={bf,footnotesize}}

%
% code listing style
%

\lstdefinestyle{kit-cm}{
  backgroundcolor=\color{light-gray},
  belowcaptionskip=1\baselineskip,
  breaklines=true,
  frame=single,
  framexleftmargin=15pt,
  language=C,
  showstringspaces=false,
  basicstyle=\footnotesize\ttfamily, 
  numbers=left,                    
  numbersep=7pt,                
  numberstyle=\tiny\color{black},
  captionpos=b
}

\lstdefinelanguage{Swift}{
  keywords={associatedtype, class, deinit, enum, extension, func, import, init, inout, internal, let, operator, private, protocol, public, static, struct, subscript, typealias, var, break, case, continue, default, defer, do, else, fallthrough, for, guard, if, in, repeat, return, switch, where, while, as, catch, dynamicType, false, is, nil, rethrows, super, self, Self, throw, throws, true, try, associativity, convenience, dynamic, didSet, final, get, infix, indirect, lazy, left, mutating, none, nonmutating, optional, override, postfix, precedence, prefix, Protocol, required, right, set, Type, unowned, weak, willSet},
  ndkeywords={class, export, boolean, throw, implements, import, this},
  sensitive=false,
  comment=[l]{//},
  morecomment=[s]{/*}{*/},
  morestring=[b]',
  morestring=[b]"
}

\lstdefinelanguage{Gherkin}{
	morekeywords = {
		Given,
		When,
		Then,
		And,
		Scenario,
		Feature,
		But,
		Background,
		Scenario Outline,
		Examples
	},
	sensitive=true,
	morecomment=[l]{\#},
	morestring=[b]",
	morestring=[b]'
}

\DeclareCaptionFont{black}{\color{black}} 
\DeclareCaptionFormat{listing}
  {\colorbox{white}
     {\parbox{\dimexpr\textwidth-2\fboxsep}{\centering #1#2#3}}}
\captionsetup[lstlisting]{format=listing,labelfont=black,textfont=black}

\usepackage{hyperref}
\usepackage{paralist}
\usepackage{subcaption}
\usepackage{todonotes}
\usepackage{booktabs}
\usepackage{listings}
\usepackage{caption}

%
% glossary
%

\usepackage[acronym,numberedsection]{glossaries}
\setacronymstyle{long-short}
\makeglossaries

%
% load additional packages
%

\usepackage{calc}
\usepackage{enumitem}
\usepackage{multirow}
\usepackage{mathtools}
\usepackage{enumitem}
\usepackage{amsmath} 
\usepackage{amssymb}
\usepackage{wasysym}
\usepackage{rotating}
\usepackage{pgfplots}
\usepackage{longtable}
\usepackage{algorithm}
\usepackage{algpseudocode}
\usepackage{float}
\usepackage{pdfpages}
\usepackage{threeparttablex}
\usepackage{longtable,lscape}
\usepackage{tablefootnote}
\usetikzlibrary{patterns}
\usepackage{multicol}
\usepackage{bm}
\usepackage{esvect}
%\usepackage{subfigure}
\floatname{algorithm}{Algorithmus}
